\documentclass[a4paper, 12pt]{extarticle}

\usepackage[utf8]{inputenc}
\usepackage[russian]{babel}

\usepackage{hyperref}
\usepackage{multirow} 
\usepackage{graphicx}
\usepackage{bm}
\usepackage{geometry}
\usepackage{listings}
\usepackage{tabularx}
\usepackage{tikz}
\usetikzlibrary{automata,positioning}

\geometry{a4paper,top=1.5cm,bottom=1.5cm,bindingoffset=0cm}
\geometry{left=3cm,textwidth=16cm}
% \linespread{1.0}
  
\usepackage{amssymb}
\usepackage{amsmath}
\usepackage{amsthm}
\usepackage{verbatim}

\title{}
\author{}
\date{}

\newtheorem*{definition}{Определение}
\newtheorem*{homework}{Упражнение}

\newcommand{\MTRule}[4]{#1 \rightarrow #2 #3 \ #4}
\newcommand{\MTRuleTwo}[7]{#1 & #2 & \rightarrow & #3#4\ &\ #5#6 & #7}
\newcommand{\AhoUlman}{[А.Ахо, Дж.Хопкрофт, Дж.Ульман. Построение и анализ вычислительных алгоритмов. М.: Мир, 1979]}


\begin{document}

\section*{Семинар \#1. Машина Тьюринга. Оценки сложности.}

\begin{definition}
 Алфавит $\Sigma$ -- конечное множество символов.
\end{definition}

\begin{definition}
 Слово $w$ -- конечная последовательность символов.
\end{definition}

\begin{definition}
 Язык $L$ -- множество слов в алфавите $\Sigma$.
\end{definition}

\begin{definition}
 k-ленточная машина Тьюринга ($MT^{k}$) -- кортеж $\langle \Sigma, Q, q_0, Q_F, \pi\rangle$.
 \begin{itemize}
  \item $Q$ -- конечное множество состояний,
  \item  $Q_F \subset Q$ -- множество заключительных состояний,
  \item  $q_0 \in Q$ -- начальное состояние,
  \item  $\pi : Q \times \Sigma^k \rightarrow Q \times (\Sigma \times (L, R, H))^k$ -- программа.
 \end{itemize}
\end{definition}

\begin{definition}
 Конфигурация МТ $\langle q, S, P\rangle$: 
 \begin{itemize}
  \item $q \in Q$  -- текущее состояние,
  \item $S$ -- состояние ленты,
  \item $P$ -- позиции головок.
 \end{itemize}
\end{definition}

\begin{definition}
 Протокол -- последовательность пройденных конфигураций.
\end{definition}

\begin{definition}
 Словарная функция $\phi_M: \Sigma^* \rightarrow \Sigma^*$.
\end{definition}

\begin{definition}
 Временная сложность в худшем случае: 
 $$T_M(n) = max\{t_M(w) \ . \ \|w\| \leq n \land w \in D(\phi_M)\}$$
\end{definition}

\begin{definition}
 Временная сложность в среднем:
  $$T_M(n) = \sum_{\|w\| = n} t_M(w) p(n, w)$$
  где, $p(n, w)$ -- вероятность появления $w$ среди всех входов длины $n$.
\end{definition}

\subsection*{Пример работы. Запись в табличном и графовом виде.}
Задача: написать программу для одноленточной МТ в алфавите $\{0, 1, \#\}$, которая инвертирует двоичную строку. Например,
входные данные $\#0011101010\#$ следует преобразовать в $\#1100010101\#$.

Запись МТ в табличном виде:
\begin{align*}
START: &                            & INVERT:&                         \\
       & \MTRule{\#}{\#}{R}{INVERT} &       & \MTRule{\#}{\#}{H}{STOP} \\
       & \MTRule{0}{\#}{H}{ERROR}   &       & \MTRule{0}{1}{R}{INVERT} \\
       & \MTRule{1}{\#}{H}{ERROR}   &       & \MTRule{1}{0}{R}{INVERT} 
\end{align*}

\begin{figure}[h!]
\begin{center}
\begin{tikzpicture}[shorten >=1pt,node distance=5cm,on grid,auto] 
   \node[state] (START)   {$START$}; 
   \node[state] (INVERT) [right=of START] {$INVERT$}; 
   \node[state, accepting] (STOP) [right=of INVERT] {$STOP$}; 
   \node[state](ERROR) [below=of START] {$ERROR$};
    \path[->] 
    (START) edge  node {$\# \rightarrow \#R$} (INVERT)
            edge  node {$\{0, 1\} \rightarrow \#H$} (ERROR)
    (INVERT) edge  node  {$\# \rightarrow \#H$} (STOP)
             edge [loop above] node {$0 \rightarrow 1R$} (INVERT)
             edge [loop below] node {$1 \rightarrow 0R$} (INVERT);
\end{tikzpicture}
\end{center}
\caption{Запись МТ в графовом виде}
\end{figure}


\subsection*{Инкремент}
Задача: написать программу для одноленточной МТ в алфавите $\{0, 1, \#\}$, которая к двоичному числу, записанному ``слева направо'' 
(т.е. наименее значимый бит -- Least Significant Bit, LSB -- слева) без лидирующих нулей, прибавляет единицу. Примеры:
$$\#1\# \Rightarrow \#01\#$$
$$\#001\# \Rightarrow \#101\#$$
$$\#1001\# \Rightarrow \#0101\#$$

Оценить сложность в худшем. Оценить сложность в среднем.

\subsubsection*{Подсказки}
Разбить множество всех входов фиксированной длины $I_n$ на классы $I^{0}_n, I^{1}_n, ..., I^{n}_n$, где
$$I^{i}_n = \{ w \in I. \ w = \# \underbrace{\underbrace{1...1}_{\text{$i$ единиц}}0\underbrace{*...*}_{\text{произвольные символы}}}_{\text{всего $n$ цифр}}\#\}$$

Найти $T_i(w)$ для $w \in I^{i}_n$. Вычислить $\vert{I^{i}_n}\rvert$. 

$$S_1(t) = t + t^2 + ... + t^n \Rightarrow tS_1(t) - S_1(t) = t^{n+1} - t \Rightarrow ...$$
$$S_2(t) = t + 2t^2 + ... + nt^n \Rightarrow tS_2(t) - S_2(t) = ...$$

\begin{homework}
 Написать программу для одноленточной МТ в алфавите $\{0, 1, \#\}$, которая разворачивает данную строку, состоящую из нулей и единиц.
 Примеры:
$$\#1\# \Rightarrow \#1\#$$
$$\#001\# \Rightarrow \#100\#$$
$$\#1101\# \Rightarrow \#1011\#$$
\end{homework}

\begin{homework}
 Написать программу для одноленточной МТ в алфавите $\{0, 1, \#\}$, которая проверяет, является ли заданная строка
 правильной записью двоичного числа (LSB -- слева).
\end{homework}

\subsection*{Прибавление константы}
Задача: написать программу для одноленточной МТ в алфавите $\{0, 1, \#\}$, которая к двоичному числу, записанному слева направо
без лидирующих нулей, прибавляет тройку. Обобщить на константу $k$.

\begin{homework}
Написать программу для одноленточной МТ в алфавите $\{0, 1, ..., m - 1, \#\}$, которая к $m$-ричному числу, 
записанному слева направо без лидирующих нулей, прибавляет константу $k$.
\end{homework}

\begin{homework}
Перевести двоичное число $x$ в $16$-ричную систему счисления. $\Sigma = \{0, 1, ..., 9, A, B, C, D, E, F, \#\}$
\end{homework}

\subsection*{Дополнительный блок: строковые алгоритмы}
Задача: написать программу для одноленточной МТ в алфавите $\{a, b, \#\}$, которая распознает язык $L = \{a^nb^n \ | n \geq 0\}$.

Оценить сложность в худшем. Ускорить программу в $k$ раз (асимптотически), увеличив число состояний, но не изменяя алфавит.

\begin{homework}
Задача: написать программу для одноленточной МТ в алфавите $\{a, b, \#\}$, которая проверяет,
есть ли во входной строке подстрока $abaababa$ (\textit{шаблон}). Предложить процедуру построения 
оптимальной программы поиска для любого заранее известного шаблона.
\end{homework}

\begin{homework}
Задача: написать программу для одноленточной МТ в алфавите $\{a, b, \#\}$, которая проверяет,
есть ли в строке $ababababbbbaaaababaabaabbba$ (\textit{база данных}) подстрока, записанная на входной ленте. 
Предложить процедуру построения оптимальной программы для любой заранее известной БД.
\end{homework}

\newpage
\section*{Семинар \#2. Многоленточная МТ.}

\subsection*{Пример работы}

Задача: инвертировать обе ленты 2-ленточной MT в алфавите $\Sigma = \{0, 1, \#\}$.

Краткая запись программы: $X, Y \in \{0, 1, \#\}$, $x, y \in \{0, 1\}$.

\begin{align*}
START: &          & & & &                              \\
       & \MTRuleTwo{\#}{\#}{\#}{R}{\#}{R}{LOOP} \\
       & \MTRuleTwo{X}{Y}{X}{H}{Y}{H}{ERROR}    \\
LOOP:  &          & & & &                              \\
       & \MTRuleTwo{x}{y}{[(x+1)\ mod\ 2]}{R}{[(y + 1)\ mod\ 2]}{R}{LOOP}    \\
       & \MTRuleTwo{x}{\#}{[(x+1)\ mod\ 2]}{R}{\#}{H}{LOOP}    \\
       & \MTRuleTwo{\#}{y}{\#}{H}{[(y + 1)\ mod\ 2]}{R}{LOOP}    \\
       & \MTRuleTwo{\#}{\#}{\#}{H}{\#}{H}{STOP}    \\       
\end{align*}

\subsection*{Сложение}
Задача: написать программу для 3-ленточной МТ в алфавите $\{0, 1, \#\}$, которая 
складывает двоичные числа (LSB слева), записанные на первой и второй ленте, а результат записывает на третью ленту.

Оценить сложность в худшем.

\begin{homework}
Решить аналогичную задачу для 2-ленточной МТ. Ответ записать на вторую ленту.
\end{homework}

\subsection*{Умножение}

Задача: написать программу для 3-ленточной МТ в алфавите $\{0, 1, \#\}$, которая 
умножает двоичные числа, записанные на первой и второй ленте, а результат записывает на третью ленту.

Оценить, как зависит сложность алгоритма от:
\begin{itemize}
 \item Формата записи числа (LSB слева vs. LSB справа);
 \item Порядка записи чисел (большее число на первой ленте vs. на второй ленте);
 \item Выбранного алгоритма (столбиком vs. крестьянский способ умножения).
\end{itemize}

\begin{homework}
Написать программу для МТ и проверить ее в эмуляторе.
\end{homework}

\newpage
\section*{Семинары \#3-4. Моделирование}
\begin{definition}
 Модель вычислений. Вычислитель. Допустимые входы.
\end{definition}

\begin{definition}
 Модель вычислений $M_1$ можно моделировать моделью вычислений $M_2$, если для любой машины $A$ в $M_1$ можно построить машину 
 $B$ в $M_2$ такую, что: 
 $$\phi_A = \phi_B \text{ при подходящих интерпретациях,}$$
 $$T_B(n) = O(p(T_A(n))) \text{ для некоторого полинома p.} $$
\end{definition}

\begin{definition}
 Неотрицательные функции $f_1(n)$, $f_2(n)$ \textbf{полиномиально связаны} тогда и только тогда, когда 
 $\exists p_1(n), p_2(n)$ -- полиномы, такие что:
 $$\forall n\ f_1(n) \leq p_1(f_2(n)) \text{ и } f_2(n) \leq p_2(f_1(n))$$
\end{definition}

\begin{definition}
 Как решать задачи на моделирование:
 \begin{itemize}
  \item Зафиксировать модель вычислений $M_1$ в языке $\Sigma_1$,
  модель вычислений $M_2$ в языке $\Sigma_2$.
  \item Указать функцию $code: \Sigma_{1}^{*} \rightarrow \Sigma_{2}^{*}$. Проанализировать сложность.
  \item Указать функцию $decode: \Sigma_{2}^{*} \rightarrow \Sigma_{1}^{*}$. Проанализировать сложность.
  \item Указать алгоритм, который по вычислителю $A \in M_1$ построит вычислитель $B \in M_2$.
  \item Обосновать $\phi_A = \phi_B$. При пошаговом моделировании удобно строить индуктивное доказательство. 
        Важно: не забыть о граничных случаях -- аварийное завершение работы и зацикливание вычислителя.
  \item Сравнить $T_A(n)$ и $T_B(n)$, $|A|$ и $|B|$.
 \end{itemize}
\end{definition}

\subsection*{Моделирование б\textit{о}льшего алфавита}
Задача: промоделировать 1-ленточную МТ с алфавитом $\Sigma_1 = \{0, 1, \#\}$, на 1-ленточной МТ с алфавитом $\Sigma_2 = \{0, 1\}$.
Рассматриваются МТ с лентой, бесконечной вправо.

Идея решения: заменить букву из $\Sigma_1$ на пару букв из $\Sigma_2$.

\begin{itemize}
 \item Определим функцию кодирования $code$:
 $$w = a_1...a_n \rightarrow b_1^0b_1^1...b_n^0b_n^1$$
 где $a_i \in \Sigma_1$ переходит в $b_i^0b_i^1, b_i^j\in \Sigma_2$ согласно правилу 
 $$\# \rightarrow 00, 0 \rightarrow 01, 1 \rightarrow 11$$
 
 \item Определим функцию декодирования как функцию, обратную $code$, полагая, что $10$ переходит в символ $0$. 
 
 \begin{homework}
 Доказать, что  функцию декодирования можно доопределить произвольным образом. 
 \end{homework} 
 
 \item Зафиксируем МТ $A = \langle \Sigma_1, Q^A, q_0^A, Q_F^A, \pi\rangle$. Пусть
  $$q: \MTRule{x}{y}{R}{q'}\ \text{где } q, q' \in Q^A, x, y \in \Sigma_1$$  
  $$code(x) = x_1x_0, code(y) = y_1y_0,\ \text{где } x_1, x_0, y_1, y_0 \in \Sigma_2$$
  В таком случае МТ $B = \langle \Sigma_2, Q^B, q_0^B, Q_F^B, \pi\rangle$ должна содержать следующие команды:
  $$ q               : \MTRule{x_1}{x_1}{R}{q^{x_1\hat{?}}} $$
  $$ q^{x_1\hat{?}}  : \MTRule{x_0}{y_0}{L}{q^{\hat{x_1}y_0}} $$
  $$ q^{\hat{x_1}y_0}: \MTRule{x_1}{y_1}{R}{q^{y_1\hat{y_0}}} $$
  $$ q^{y_1\hat{y_0}}: \MTRule{y_0}{y_0}{R}{q'} $$
  
  \begin{homework}
  Указать, как трансформируются команды
  $$q: \MTRule{x}{y}{L}{q'}$$
  $$q: \MTRule{x}{y}{H}{q'}$$   
  \end{homework}
 
  \item Предположим, что на ленте МТ $A$ записано слово $w$, текущее состояние $q$, указатель находится на позиции $k$ и 
  будет выполнена команда перехода $t$, изменяющая слово на $w'$, сдвигающая указатель на позицию $k'$ и переводящая МТ в состояние $q'$.
  Предположим, что МТ $B$ была получена из $A$ по алгоритму, указанному выше. Пусть на ленте $B$ записано слово $v = code(w)$,
  текущее состояние $q$, указатель находится на позиции $n = 2 * k$ и будут выполнены команды
  перехода $t_1, t_2, t_3, t_4$, изменяющие слово на $v'$, сдвигающие указатель на позицию $n'$ и переводящие МТ в состояние $\hat{q}$.

  \begin{homework}
    Доказать, что $decode(v') = w', n' = 2 * k', \hat{q} = q'$. Рассмотреть случай $k' = -1$.
  \end{homework}
 
 \begin{homework}
    Доказать, что $\phi_A = \phi_B$. Рассмотреть случай зацикливания $A$.
  \end{homework}
   
  \item По построению 
  $$T_B(w) \leq 4 T_A(w)$$
  $$|Q_B| \leq 4 |Q_A| $$
\end{itemize}

\begin{homework}
Решить обратную задачу: промоделировать 1-ленточную МТ с алфавитом $\Sigma_1 = \{0, 1\}$, на 1-ленточной МТ с алфавитом $\Sigma_2 = \{0, 1, 2, \#\}$.
Можно ли ускорить произвольную программу?
\end{homework}

\subsection*{Задачи для моделирования}
\begin{itemize}
 \item Однонапраленная vs. двунаправленная лента.
 \item 1 лента vs. $k$ лент.
 \item $\Sigma_1$ vs. $\Sigma_2$ для $|\Sigma_1| = n, |\Sigma_2| = m, n > m$.
 \item Один указатель на текущий символ vs. $k$ указателей. 
\end{itemize}

\newpage
\section*{Семинар \#5. Конечные автоматы.}

\begin{definition}
Детерминированный конечный автомат (Deterministic Finite State Machine) / Недетерминированный конечный автомат (Nondeterministic Finite Automaton) -- 
$(X, Q, q_0, Q_F, \pi)$:
\begin{itemize}
 \item $X = \{a_1, ..., a_n\}$ -- конечный алфавит,
 \item $Q$ –  конечное множество состояний,
 \item $q_0\in Q$ – начальное состояние,
 \item $Q_F \in Q$ – множество заключительных состояний, 
 \item $\pi : Q \times X \rightarrow Q$ / $\pi : Q \times X \rightarrow 2^Q$ -- функция перехода.
\end{itemize}
\end{definition}

\begin{definition}
Регулярное выражение над $X=\{a_1, ... ,a_k\}$:
\begin{itemize}
 \item $\emptyset, \epsilon, a_i$ -- регулярные выражения;
 \item если $\alpha, \beta$ -- регулярные выражения, то
 \begin{itemize}
  \item $\alpha\beta$ -- конкатенация,
  \item $\alpha|\beta$ -- альтернатива,
  \item $(\alpha)^*$ -- итерация;
 \end{itemize}
 \item других нет.
\end{itemize}
\end{definition}

\begin{definition}
 Основные теоремы:
 \begin{itemize}
  \item Любой язык, распознаваемый НКА, может быть распознан некоторым ДКА.
  \item Любой язык, распознаваемый ДКА, соответствует языку, задаваемому некоторым регулярным выражением.
  \item Любой язык, задаваемый регулярным выражением, может быть распознан некоторым НКА.
 \end{itemize}
\end{definition}

\subsection*{Пример}
Язык $L = \{\text{слово в алфавите }\Sigma = \{0, 1\},\text{ которое оканчивается на }01\}$.

Регулярное выражение - $(0|1)^*01$.

\begin{figure}[h!!]
\resizebox{\textwidth}{!}{%
\begin{tabular}{cc}
ДКА & НКА            \\
 \begin{tikzpicture}[-stealth,shorten >=2pt,node distance=3cm,auto] 
 \node[state, initial] (q1) {};
 \node[state, right of=q1] (q2) {};
 \node[state, accepting, right of=q2] (q3) {};
\draw
 (q1) edge[loop below] node {1} (q1)
 (q1) edge node{0} (q2)
 (q2) edge[loop above] node {0} (q2)
 (q2) edge node{1} (q3)
 (q3) edge[bend left, below] node  {1} (q1)
 (q3) edge[bend right, above] node {0} (q2);
\end{tikzpicture}
  &  
\begin{tikzpicture}[-stealth,shorten >=2pt,node distance=3cm,auto] 
   \node[state, initial] (A)   {}; 
   \node[state] (B) [right=of A] {}; 
   \node[state, accepting] (C) [right=of B] {};    
    \path[->] 
    (A) edge [loop above] node {$0, 1$} (A)
        edge  node {$0$} (B)
    (B) edge  node  {$1$} (C);
\end{tikzpicture}  
  \\
\end{tabular}
}
\end{figure}

\subsection*{Табличный метод проверки на эквивалентность ДКА}
См. лекции.

\subsection*{Минимизация ДКА методом грубейшего разбиения}
\AhoUlman

Cтраницы 181 - 187 (пункт 4.13 ``Разбиение'').

\subsection*{Проверка эквивалентности ДКА алгоритмом Объединить-Найти (Union-Find)}
\AhoUlman

Страницы 165 - 168 (приложение 3 ``Эквивалентность конечных автоматов'').

\subsection*{Автоматы для разбора на семинаре}

\subsubsection*{I}
\begin{tikzpicture}[-stealth,shorten >=2pt,node distance=3cm,auto] 
  \tikzstyle{every state}=[fill=none,draw=black,text=black]
      \node[initial,accepting,state]           (1)              {1};     
      \node[accepting,state]         (2) [right of=1] {2};     
      \node[accepting,state]         (3) [below of=1] {3};   
      \node[state]                   (4) [right of=3] {4};                    
  \path 
	  (1) edge node {0} (2)
	  (1) edge node {1} (3)
	  
	  (2) edge [loop above] node {0} (2)
	  (2) edge [bend right] node {1} (3)
	  
	  (3) edge node {1} (4)
	  (3) edge [bend right] node {0} (2)
	  
	  (4) edge [loop right] node {0, 1} (4)
	  ;	
\end{tikzpicture}

\subsubsection*{II}
\begin{tikzpicture}[-stealth,shorten >=2pt,node distance=3cm,auto] 
  \tikzstyle{every state}=[fill=none,draw=black,text=black]
      \node[initial,accepting,state] (1)              {1};     
      \node[accepting,state]         (2) [right of=1] {2};     
      \node[accepting,state]         (3) [below of=1] {3};
      \node[accepting,state]         (4) [below of=2] {4};   
      \node[state]                   (5) [below of=3] {5};         
      \node[state]                   (6) [below of=4] {6};                    
  \path 
	  (1) edge node {0} (2)
	  (1) edge node {1} (3)
	  
	  (2) edge [loop above] node {0} (2)
	  (2) edge [bend right] node {1} (4)
	  
	  (3) edge node {0} (2)
	  (3) edge node {1} (5)
	  
	  (4) edge [bend right] node {0} (2)
	  (4) edge node {1} (6)
	  
	  (5) edge [loop right] node {0, 1} (5)
	  (6) edge [loop right] node {0, 1} (6)
	  ;	
\end{tikzpicture}

\newpage
\section*{Семинар \#6. RAM машина.}

\begin{figure}[h]
\begin{tabular}{ccc | ccc}
Команда & Аргумент & Символьная запись & Команда & Аргумент & Символьная запись\\
0  &  0     & Halt 0  &                      13 &  13    & Mult 13 \\           
1  &  1     & Load =1  &                     14 &  14    & Mult *14 \\            
2  &  2     & Load 2  &                      15 &  15    & Div =15 \\           
3  &  3     & Load *3  &                     16 &  16    & Div 16 \\            
4  &  4     & Store 4  &                     17 &  17    & Div *17 \\            
5  &  5     & Store *5  &                    18 &  18    & Read 18 \\             
6  &  6     & Add =6  &                      19 &  19    & Read *19 \\           
7  &  7     & Add 7  &                       20 &  20    & Write =20 \\          
8  &  8     & Add *8  &                      21 &  21    & Write 21 \\           
9  &  9     & Sub =9  &                      22 &  22    & Write *22 \\           
10 &  10    & Sub 10  &                      23 &  23    & Jump  23   \\           
11 &  11    & Sub *11  &                     24 &  24    & JZero  24   \\            
12 &  12    & Mult =12  &                    25 &  25    & JGTZ  25                
\end{tabular}
\end{figure}

\begin{definition}
Словарная функция.
\end{definition}

\begin{definition}
Равномерный весовой критерий.
\end{definition}

\begin{definition}
Логарифмический весовой критерий.
\end{definition}

\subsection*{Пример программы}
\begin{verbatim}
          Read 1                ; R1 = N, R2 = 0, R0 = 0
          
LOOP:     Load 2                ; R0 = R2
          Mult 0                ; R0 = R0 * R0
          Sub 1                 ; R0 = R0 - R1
          JZero FINISHED        ; if R0 == 0 goto FINISHED
          JGTZ FIN_SUB          ; if R0 > 0  goto FIN_SUB
          Load 2                ; R0 = R2
          Add =1                ; R0 = R0 + 1
          Store 2               ; R2 = R0
          JUMP LOOP             ; goto LOOP

FIN_SUB:  Load 2                
          Sub =1
          Store 2               ; R2 = R2 - 1

FINISHED: Load =2               ; R0 = 2
          Write *0              ; print (R_(R_0)) ~ print(R_2) ~ print(R2)
          Halt 0                ; stop
\end{verbatim}

\newpage
\subsection*{Алгоритм Евклида: пошаговое кодирование}

\newcommand{\TabWidth}{2in}
\begin{tabular}{lll}
 Псевдокод & Циклы -> метки & Условия -> метки\\
 \hline 
 & & \\
 \begin{minipage}{1.5in} 
\begin{verbatim}
 a <- read()
 b <- read()
 
 while (a != b) {
   if (a > b) {
   
   
   
     a = a - b
     
   } else {
      
     b = b - a
     
   }
   
 }
 
  
 write(a)
 
\end{verbatim}
\end{minipage}
&
\begin{minipage}{\TabWidth} 
\begin{verbatim}
 a <- read()
 b <- read() 
 LOOP:
   if (a == b) goto END
   if (a > b) {
   
   
      
     a = a - b
     
   } else {
   
     b = b - a
     
   } 
   goto LOOP

 END:
  
   write(a)
   
\end{verbatim}
\end{minipage}
&
\begin{minipage}{\TabWidth} 
\begin{verbatim}
 a <- read()
 b <- read() 
 LOOP:
   if (a == b) goto END
   if (a > b) goto SUB_A
   goto SUB_B
   
SUB_A:
   a = a - b
   goto AFTER_IF:  
   
SUB_B:
   b = b - a
   goto AFTER_IF
   
AFTER_IF:   
   goto LOOP
 
END: 
   write(a)
   
\end{verbatim}
\end{minipage} \\
\hline
\end{tabular}



\noindent \begin{tabular}{lll}
 Удаление лишних переходов & Распределение регистров & Оптимизация общих выражений \\
 \hline 
 & & \\
\begin{minipage}{1.5in} 
\begin{verbatim}
 a <- read()
 b <- read() 
 LOOP:
 
   if (a == b) goto END
   if (a > b) goto SUB_A

   b = b - a
   
   goto LOOP
   
SUB_A:
   a = a - b
   goto LOOP  
       
END: 
   write(a)
\end{verbatim}
\end{minipage}
&
\begin{minipage}{\TabWidth} 
\begin{verbatim}
 R1 <- read()
 R2 <- read() 
 LOOP:
   R0 = R1 - R2
   if (R0 == 0) goto END
   if (R0 > 0) goto SUB_A

   R2 = R2 - R1
   
   goto LOOP
   
SUB_A:
   R1 = R1 - R2
   goto LOOP  
       
END: 
   write(R1)
\end{verbatim}
\end{minipage}
&
\begin{minipage}{\TabWidth} 
\begin{verbatim}
 R1 <- read()
 R2 <- read() 
 LOOP:
   R0 = R1 - R2
   if (R0 == 0) goto END
   if (R0 > 0) goto SUB_A

   R0 = -R0
   R2 = R0
   goto LOOP
   
SUB_A:
   R1 = R0
   goto LOOP  
       
END: 
   write(R1)
\end{verbatim}
\end{minipage}
\end{tabular}



\begin{tabular}{lll}
 RAM ассемблер & Машинный код \\
 \hline 
 &  \\
\begin{minipage}{\TabWidth} 
\begin{verbatim}
1.         Read 1
2.         Read 2
3. LOOP:   Load 1
4.         Sub 2
5.         JZero END
6.         JGTZ SUB_A
7.         MULT =-1
8.         Store 2
9.         Jump LOOP
10. SUB_A: Store 1
11.        Jump LOOP
12. END:   Write 1
13.        Halt 0
\end{verbatim}
\end{minipage}
&
\begin{minipage}{\TabWidth} 
\begin{verbatim}
18   1
18   2            
2    1      
10   2      
24   12     
25   10     
12   -1     
4    2      
23   3      
4    1      
23   3      
21   1      
0    0      
\end{verbatim}
\end{minipage}
\end{tabular}

\begin{homework}
 Провести анализ сложности алгоритма при логарифмическом весовом критерии.
\end{homework}

\subsection*{Косвенная адресация}
Задача: написать программу для RAM-машины, которая вычисляет скалярное произведение двух непустых векторов.

Вход: $n \geq 1$, $a_1, ..., a_n, b_1, ..., b_n$

Выход: $\sum_{i=1}^{n} a_i b_i$

Проанализировать сложность программы, используя равномерный и логарифмический весовой критерий.

\subsection*{Реализация стека}
Задача: написать программу для RAM-машины, которая выводит вектор целых чисел в обратном порядке. Признак конца вектора -- число $0$.

Вход: $a_1, ..., a_n, 0$, где $a_i \neq 0$.

Выход: $a_n, a_{n-1}, ..., a_1$

Использовать псевдопроцедуры $push, pop$.

\newpage
\section*{Семинар \#7. RAM машина как ассемблер ЯВУ: вызов функций}

\subsection*{Функция высшего порядка map}
Задача: написать программу для RAM-машины, которая применяет некоторую целочисленную функцию $f$ к массиву чисел.

Вход: $n > 0, a_1, ..., a_n$.

Выход: $f(a_1), ..., f(a_n)$.

\noindent Обобщить на функции от нескольких параметров:

Вход: $n > 0, 0 < k < 5, a_1^1, a_1^2, ..., a_1^k, ..., a_n^1, a_n^2, ..., a_n^k$.

Выход: $f(a_1^1, a_1^2, ..., a_1^k), ..., f(a_n^1, a_n^2, ..., a_n^k)$.

\subsection*{Соглашение о вызовах и адрес возврата}
Переписать программу из предыдущего пункта, используя следующий контракт:
\begin{itemize}
 \item В регистре $R_1$ передается номер инструкции, на которую следует перейти после выполнения $f$ (ret-address).
 \item В регистрах $R_2, ..., R_5$ передаются параметры для $f$ (param-passing regs). После выполнения $f$ регистры должны содержать то же самое значение (non-volatile regs).
 \item Регистры $R_6, ..., R_{10}$ после выполнения $f$ могут произвольным образом измениться (volatile regs).
 \item В регистре $R_0$ должен быть сохранен результат вычисления функции.
\end{itemize}

Реализовать ``псевдоинструкцию'' $call$.

\begin{homework}
 Реализовать транслятор ``расширенного'' RAM-ассемблера, использующего инструкцию $call$, в обычный ассемблер.
 Использовать любой язык программирования. Возможно ли таким образом реализовать рекурсивные вызовы?
\end{homework}

\subsection*{Дополнительный блок: арифметические выражения}
\begin{homework}
 Дана строка в обратной польской записи, в которой встречаются только положительные целые числа и коды операций:
 $$+ \rightarrow -1$$
 $$- \rightarrow -2$$
 $$* \rightarrow -3$$
 $$/ \rightarrow -4$$
 
 Вычислить данное выражение либо зациклиться, если происходит недопустимая операция (деление на ноль).
\end{homework}

\begin{homework} 
 Усложнение предыдущей задачи: в строке встречаются переменные
 $$x \rightarrow -5$$
 $$y \rightarrow -6$$
 $$z \rightarrow -7$$
 и задана таблица их значений в виде пар $\langle$ код переменной, значение переменной $\rangle$.
\end{homework}

\newpage
\section*{Семинары \#8-9. RASP машина. Самомодифицирующийся код. JIT компиляция. Эмуляция косвенности. Загрузчик. Квайн. }

\subsection*{Пример самомодифицирующейся программы}
\begin{verbatim}
START: Load  =0
       Store  9
       Load  =0
       Store 10
       Jump  START
\end{verbatim}

\subsection*{Переход на произвольную инструкцию}
\begin{verbatim}
LABEL: Load x            ; R0 = x
       Store [LABEL + 5] ; --|
       JUMP Label        ; <-|  JUMP x
\end{verbatim}

\begin{homework}
  Объяснить, как реализовать рекурсивный вызов процедур на RASP машине.
  Использовать концепцию стекового кадра и стека вызовов\footnote{\url{https://en.wikipedia.org/wiki/Call_stack}}.
\end{homework}

\begin{homework}
 Реализовать транслятор ``расширенного'' RASP-ассемблера, использующего инструкцию $call$, в обычный ассемблер.
 Поддержать рекурсивные вычисления, в том числе косвенную рекурсию. Использовать любой язык программирования. 
\end{homework}

\subsection*{Генерация программы в памяти во время исполнения}
Задача: написать \textit{процедуру} для RASP-машины, которая принимает аргументом число $n > 0$ и печатает числа от $1$ до $n$ включительно. 

\begin{homework}
Оценить алгоритмическую сложность программы при равномерном весовом критерии.
\begin{verbatim}
void printer(int n) {
  i = 0
LOOP: 
  write (i)
  if (n == 0) goto END
  i = i + 1
  n = n - 1
  goto LOOP
END: 
  return 
}   
\end{verbatim}
\end{homework}

\subsubsection*{Loop unrolling}
 Оценить алгоритмическую сложность программ при равномерном весовом критерии.

\begin{tabular}{l|l}
& \\
\begin{minipage}{3in} 
\begin{verbatim}
void printer_2_naive(int n) {
  i = 0
LOOP: 
  write (i)
  if (n == 0) goto END
  i = i + 1
  n = n - 1

  write (i)
  if (n == 0) goto END
  i = i + 1
  n = n - 1

  goto LOOP
END: 
  return 
}   



\end{verbatim}
\end{minipage}
&
\begin{minipage}{3in} 
\begin{verbatim}
void printer_2_preheader(int n) {
  i = 0

  mod = n % 2
  if (mod == 0) goto LOOP  
  write (i)
  i = i + 1
  n = n - 1
     
LOOP: 
  write (i)
  i = i + 1
  write (i)
  i = i + 1  
  n = n - 2
  if (n == 0) goto END  
  goto LOOP
END: 
  return 
}   
\end{verbatim}
\end{minipage}
\end{tabular}

\begin{homework}
 Напиcать RASP-программу, которая для заданного $k$ генерирует программу $printer\_k\_preheader$.
 Оценить сложность программы генерации и сложность сгенерированной программы.
\end{homework}

\begin{homework}
 Проанализировать эффективность оптимизации ``Раскрутка цикла'' -- найти формулы для оценки сложности при равномерном весовом критерии
 и для затрат памяти. Какие стратегии раскрутки кажутся Вам наиболее эффективными? Привести контрпримеры.
 Можно ли использовать информацию времени выполнения (a.k.a. execution profile), чтобы принять оптимальное решение?
\end{homework}

\subsection*{Загрузчик программ. Отличие от интерпретатора.}
Задача: написать RASP-программу, которая считывает с входной ленты поток чисел длины $N$, являющийся допустимой RASP-программой,
располагает ``входную программу'' в регистрах $R_1, R_2, ..., R_N$, а затем передает ей управление, выполнив $JUMP$ на регистр $R_1$.

\subsection*{Вопросы для самостоятельного разбора.}
Итоговая контрольная работа содержит задачи на материал данного раздела.

\subsubsection*{Пример RASP программы, выводящей свой собственный код.}
\url{https://en.wikipedia.org/wiki/Quine_(computing)}

\subsubsection*{Эмуляция косвенности (операций со звездочкой).}
\AhoUlman

Cтраницы 26 - 31 (пункт 1.4 ``Модель с хранимой программой'').
\end{document}
