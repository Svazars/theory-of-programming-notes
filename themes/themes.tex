\documentclass[a4paper, 12pt]{extarticle}

\usepackage[utf8]{inputenc}
\usepackage[russian]{babel}

\usepackage{multirow} 
\usepackage{graphicx}
\usepackage{bm}
\usepackage{geometry}

\geometry{a4paper,top=2cm,bottom=2cm,bindingoffset=0cm}

% <leftmargin> | <text> | <rightmargin>
%     3cm      |  16cm  |     2cm
\geometry{left=3cm,textwidth=16cm}
\linespread{1.3}
  
\usepackage{amssymb}
\usepackage{amsmath}

\title{Темы докладов}
\author{Филатов А. Ю.}
\date{}

\begin{document}

\section*{Литература}

\begin{itemize}
 \item АХУ: А. Ахо, Дж. Хопкрофт, Дж. Ульман, Построение и анализ вычислительных алгоритмов, М.: Мир, 1979
 \item Котов: В. Е. Котов. Введение в теорию схем программ,  Новосибирское отделение издательства <<Наука>>, 1978
\end{itemize}


\section*{Темы докладов}

\begin{itemize}
 \item АХУ 3.2, с 95-104. "Цифровая сортировка."
 \item АХУ 3.6-3.7. "Порядковые статистики. Среднее время для порядковых статистик." 
 
 \item Котов. Глава 2, раздел 2. "Многоленточные и многоголовочные автоматы."
 \item АХУ. "Древовидные структуры для задачи объединить-найти."
 
 \item АХУ 4.5. "Оптимальные деревья двоичного поиска."
 \item Котов. Глава 6, разделы 1-2. "Рекурсивные схемы. Проблемы трансляции."
\end{itemize}

\thispagestyle{empty}
\end{document}
